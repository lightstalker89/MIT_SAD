%!TEX root = _Daba.tex
\chapter{Car2Car Communication}
\label{cha:Car2Car}
C2C describes the communication between vehicles and other infrastructure. The goal is to improve the safety on the streets and to inform road users about upcoming problems on the road immediately including different car manufacturers and roadside units. Furthermore the C2C Communication technology should be a basis for decentralized active safety applications and therefore reduce accidents and their severity. Besides active safety functions, it includes active traffic management applications and helps to improve traffic flow.

\section{Actors}
\label{sec:Actors}
One Actor of the System is the driver, who receives road information and warning messages or route recommendations.\\
Another Actor is the road operator, which receives road information from cars or other infrastructure and therefore will improve the control of the traffic in a more efficient way.\\
The last important actors are hotspot and internet providers, who can install their communication systems for example at gas stations.

\section{Car 2 Car Communication Safety Scenarios}
\label{sec:C2CSafetyScenarios}
\begin{description}
  \item[Cooperative forward collision warning:] \hfill \\ This scenario should avoid rear-end collisions, for example if a following vehicle suddenly brakes. The vehicles share information about speed, position and heading. To avoid collisions, the system has to use the own vehicle information and the information of vehicles nearby. If the system detects a critical proximity, it will warn the driver.
  \item[Pre-crash Sensing/Warning:] \hfill \\ If a crash is unavoidable, information will be provided about vehicle size and exact position. Crash involved vehicles will exchange data about predicted impact zones, therefore airbags or bumper systems will be informed, where the impact takes place. 
  \item[Hazardous Location Notification:] \hfill \\ The vehicle will inform about hazardous road conditions. If, for example, the ESP (Electronic Stability Program) is activated, the location and road condition will be transmitted to nearby vehicles. This information could be used for optimizing the chassis of the vehicle if it reaches the hazardous location. Such information is not limited to vehicles. Road signs could provide information over a token system, which will be served by external service providers. 
\end{description}	

\section{Car 2 Car Communication Traffic Efficiency}
\label{sec:C2CTarfficEfficiency}
\begin{description}
  \item[Enhanced route guidance and navigation:] \hfill \\ Every car should have internet access, which will be used for enhanced route guidance and navigation. For example, if no vehicle or roadside unit is ahead, road information can be provided by the internet connection. Because of the navigation system the car knows exactly where it will go. With this information data about the route can be downloaded an displayed to the driver. 
  \item[Green light optimal speed advisory:] \hfill \\ This Scenario should help the driver to make their driving smoother and avoid stopping. The information will be provided by signal intersections. The timing (when turns the light green) and exact location of the intersection will be transmitted. With this information, the vehicle calculates an optimal vehicle speed using the distance from the vehicle to the intersection and the time when the signal is green. The vehicle notifies the driver of the optimal speed. It is the goal to increase traffic flow and to increase fuel economy. 
  \item[Merging Assistance:] \hfill \\ If the vehicle wants to merge into traffic on a roadway, nearby vehicles will be informed about the approaching vehicle. The vehicle itself receives information about the current behavior of nearby vehicles. The assistance will guarantee that the vehicle can enter the traffic flow without major disruptions to the flow. 
\end{description}	

\section{Car 2 Car Communication Infotainment and Other Services}
\label{sec:C2COtherServices}
\begin{description}
  \item[Internet access in vehicle:] \hfill \\ This scenario should avoid rear-end collisions, for example if a following vehicle suddenly brakes. The vehicles share information about speed, position and heading. To avoid collisions, the system has to use the own vehicle information and the information of vehicles nearby. If the system detects a critical proximity, it will warn the driver.
  \item[Point of interest information:] \hfill \\ The Point of Interest Notification allows local businesses, tourist attractions, or other points of interest to advertise their availability to nearby vehicles. In this case a roadside unit broadcasts information about opening hours or prices. The information will only be shown to the driver in appropriate situations. For example, if the fuel is running low, the vehicle presents the driver information about nearby gas stations.
  \item[Remote Diagnostics:] \hfill \\ Remote diagnostic allows service stations to assess the state of the vehicle without a making physical connection. This would allow software updates directly to the car, without the need to drive to a service stations. When a vehicle enters the area of a service garage, the 
service garage can query the vehicle for its diagnostic information to support the diagnosis of the problem reported by the customer. Furthermore the vehicles' past history and the customers' information can be loaded from a database to support the technician. With remote diagnostics, the time in service garages will be reduced and it will also result in lower cost for repair. 

\end{description}	