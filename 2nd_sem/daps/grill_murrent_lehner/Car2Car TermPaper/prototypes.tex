\chapter{Prototypes}
\label{cha:Prototypes}
\section{Android}
Since Android 4.0, devices with appropriate hardware are allowed to connect directly to each other over WI-FI P2P without an access point between them. Android P2P framework complies with the WI-FI Alliances' WI-FI Direct certification program. With the usage of this API you are able to discover and connect to other devices when they support WI-FI P2P.  According to documentations the advantage of WI-FI P2P beside Bluetooth or similar connection types is a fast connection across distances much longer than others. This allows applications a fast exchange of data between multiple users, which could be useful for applications such as multiplayer games, photosharing applications and in general, all applications which are relying on a fast connection between a long distance.

\subsection*{Android Prototype}
\label{subsec:AndroidPrototype}
In regard to the Car2Car project an Android application which tests the reliability and the functions of the WI-FI P2P APIs was developed. In light of the idea behind the Car2Car project and the ability of modern Android phones, to track the location of a user, this subchapter will show the results of the simple WI-FI P2P and GPS prototype.
The simple prototype should discover available peers, after a successful connection it should send the GPS location of the user to all connected peers. All peers should mark the position of the other devices on the included google maps map with a marker. The picture below shows the design of the prototype application and describes the different sections.

\subsection*{Limitations and Problems}
\label{subsec:LimitationsProblems}
According to the WI-FI P2P documentation the range of the WI-FI P2P signal could be up to 500 meters. For the Android Car2Car prototype the test devices Samsung Galaxy S4 and Samsung Galaxy S2 Plus was used. It was not possible to confirm the range of 500 meters with the two devices. To test the maximal range of the signal, a few tests on a straight level road were carried out. It was determined that the signal at about 100-120 meters is lost. The tests were performed on foot and by car without any major differences. For a successful reconnect the distance between the devices was about 50-70 meters. The pictures below show the performed tests and describe their results.

\section{Rasberry Pi}
\label{sec:RasberryPi}

\section{Windows Phone}
\label{sec:WindowsPhone}
Microsoft included Wifi Direct in his new Windows Phone 8.1 SDK, but actually there is no good documentation or sample which describes the usage of Wifi Direct in a Windows Phone app.\\
The other option would be to use there own namespace which connects two phones directly to each other, but this requires Bluetooth and WIFI and the same app on both devices. Since this is not compatible with any other devices than Windows Phones this is not good solution. Furthermore are the devices limited to the Bluetooth range which is in fact not very long.