\chapter{Prototypes}
\label{cha:Prototypes}
\section{Android}
Since Android 4.0, devices with appropriate hardware are allowed to connect directly to each other over WI-FI P2P without an access point between them. Android P2P framework complies with the WI-FI Alliances' WI-FI Direct certification program. With the usage of this API you are able to discover and connect to other devices when they support WI-FI P2P.  According to documentations the advantage of WI-FI P2P beside Bluetooth or similar connection types is a fast connection across distances much longer than others. This allows applications a fast exchange of data between multiple users, which could be useful for applications such as multiplayer games, photosharing applications and in general, all applications which are relying on a fast connection between a long distance.

\section{Rasberry Pi}

\section{Windows Phone}
Microsoft included Wifi Direct in his new Windows Phone 8.1 SDK, but actually there is no good documentation or sample which describes the usage of Wifi Direct in a Windows Phone app.\\
The other option would be to use there own namespace which connects two phones directly to each other, but this requires Bluetooth and WIFI and the same app on both devices. Since this is not compatible with any other devices than Windows Phones this is not good solution. Furthermore are the devices limited to the Bluetooth range which is in fact not very long.