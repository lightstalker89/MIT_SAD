%!TEX root = _DaBa.tex
\chapter{Conclusion and future work}
\label{cha:ConclusionFutureWork}
This chapter will recap on the goals of this paper and will present future work and changes in the field of car2car communication.

\section{Conclusion}
\label{sec:Conclusion}
As a conclusion it can be said, that wifi direct is at the moment not the ideal solution to bring car2car communication to the public. The problem is, that most manufacturers have their own protocol for this technologie. iOS has the MultiPeerFramework, which can be used only for iOS Devices. Windows Phone introduced Wifi direct within the Windows Phone 8.1 SDK release, which is at the moment only available for developers, however no suitable documentation was found. Android has introduced Wifi direct since Android 4.0 and obviously uses the standard protocol, because a connection could be established between the Raspberry Pi and an Android Device. Speaking of the Rapsberry Pi the current implementation for supporting wifi direct is not reliable enough at the moment to be used in such field as car2car communication. In some tests the connection was aborted all of a sudden. This technologie needs to be more mature and consistent protocols or implementations among the smartphone manufacturers are needed.

\section{Future work}
\label{sec:FutureWork}
If the implementation of the prototype will be continued, the next step would be to concentrate on the Raspberry Pi platform. Further tests should be processed and a deeper investigation of the wifi direct/p2p protocol would be necessary. If wifi direct will not become more reliable in the future, it would be possible to use a different technologie, like communication over light. Breaking lights could be used, however it depends on free sight to the lights of the man in front and back. At the moment the prototype implementation is able to communicate with one device, later it should be possible to broadcast messages and information to more than one device in the near surroundings. Recent news show that car2car communication will be more and more important for the information technologie. INTEL for example, recently launched the automotive platform "Kendrick Peak", which should be used in in-vehicle-infotainment systems, based on an embedded atom chipset.\footnote{http://www.golem.de/news/intel-kendrick-peak-automotive-plattform-mit-atom-antrieb-im-kofferraum-1405-106821.html} This platform could be used in the field of car2car communication too. At the moment INTEL works with car manufacturers like BMW, Jaguar, Landrover, Kia and Hyundai on this platform. A cheap and interesting solution for car2car communication could be the "Vehicle Pi".\footnote{http://vehiclepi.com/} It is a shield specially designed for the Raspberry Pi, which offers sensors and interfaces that can be connected to the car and retrieve information from it. 