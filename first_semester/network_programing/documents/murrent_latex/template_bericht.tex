%%% File-Information {{{
%%% Filename: template_bericht.tex
%%% Purpose: lab report, technical report, project report
%%% Time-stamp: <2004-06-30 18:19:32 mp>
%%% Authors: The LaTeX@TUG-Team [http://latex.tugraz.at/]:
%%%          Karl Voit (vk), Michael Prokop (mp), Stefan Sollerer (ss)
%%% History:
%%%   20050914 (ss) correction of "backref=true" to "backref" due to hyperref documentation
%%%   20040630 (mp) added comments to foldmethod at end of file
%%%   20040625 (vk,ss) initial version
%%%
%%% Notes:
%%%
%%%
%%%
%%% }}}
%%%%%%%%%%%%%%%%%%%%%%%%%%%%%%%%%%%%%%%%%%%%%%%%%%%%%%%%%%%%%%%%%%%%%%%%%%%%%%%%
%%% main document {{{

\documentclass[
a4paper,     %% defines the paper size: a4paper (default), a5paper, letterpaper, ...
% landscape,   %% sets the orientation to landscape
% twoside,     %% changes to a two-page-layout (alternatively: oneside)
% twocolumn,   %% changes to a two-column-layout
% headsepline, %% add a horizontal line below the column title
% footsepline, %% add a horizontal line above the page footer
% titlepage,   %% only the titlepage (using titlepage-environment) appears on the first page (alternatively: notitlepage)
% parskip,     %% insert an empty line between two paragraphs (alternatively: halfparskip, ...)
% leqno,       %% equation numbers left (instead of right)
% fleqn,       %% equation left-justified (instead of centered)
% tablecaptionabove, %% captions of tables are above the tables (alternatively: tablecaptionbelow)
% draft,       %% produce only a draft version (mark lines that need manual edition and don't show graphics)
% 10pt         %% set default font size to 10 point
% 11pt         %% set default font size to 11 point
12pt         %% set default font size to 12 point
]{scrartcl}  %% article, see KOMA documentation (scrguide.dvi)



%%%%%%%%%%%%%%%%%%%%%%%%%%%%%%%%%%%%%%%%%%%%%%%%%%%%%%%%%%%%%%%%%%%%%%%%%%%%%%%%
%%%
%%% packages
%%%

%%%
%%% encoding and language set
%%%

%%% ngerman: language set to new-german
\usepackage{ngerman}

%%% babel: language set (can cause some conflicts with package ngerman)
%%%        use it only for multi-language documents or non-german ones
%\usepackage[ngerman]{babel}

%%% inputenc: coding of german special characters
\usepackage[latin1]{inputenc}

%%% fontenc, ae, aecompl: coding of characters in PDF documents
\usepackage[T1]{fontenc}
\usepackage{ae,aecompl}

%%%
%%% technical packages
%%%

%%% amsmath, amssymb, amstext: support for mathematics
%\usepackage{amsmath,amssymb,amstext}

%%% psfrag: replace PostScript fonts
\usepackage{psfrag}

%%% listings: include programming code
%\usepackage{listings}

%%% units: technical units
%\usepackage{units}

%%%
%%% layout
%%%

%%% scrpage2: KOMA heading and footer
%%% Note: if you don't use this package, please remove 
%%%       \pagestyle{scrheadings} and corresponding settings
%%%       below too.
\usepackage[automark]{scrpage2}


%%%
%%% PDF
%%%

\usepackage{ifpdf}

%%% Should be LAST usepackage-call!
%%% For docu on that, see reference on package ``hyperref''
\ifpdf%   (definitions for using pdflatex instead of latex)

  %%% graphicx: support for graphics
  \usepackage[pdftex]{graphicx}

  \pdfcompresslevel=9

  %%% hyperref (hyperlinks in PDF): for more options or more detailed
  %%%          explanations, see the documentation of the hyperref-package
  \usepackage[%
    %%% general options
    pdftex=true,      %% sets up hyperref for use with the pdftex program
    %plainpages=false, %% set it to false, if pdflatex complains: ``destination with same identifier already exists''
    %
    %%% extension options
    backref,      %% adds a backlink text to the end of each item in the bibliography
    pagebackref=false, %% if true, creates backward references as a list of page numbers in the bibliography
    colorlinks=true,   %% turn on colored links (true is better for on-screen reading, false is better for printout versions)
    %
    %%% PDF-specific display options
    bookmarks=true,          %% if true, generate PDF bookmarks (requires two passes of pdflatex)
    bookmarksopen=false,     %% if true, show all PDF bookmarks expanded
    bookmarksnumbered=false, %% if true, add the section numbers to the bookmarks
    %pdfstartpage={1},        %% determines, on which page the PDF file is opened
    pdfpagemode=None         %% None, UseOutlines (=show bookmarks), UseThumbs (show thumbnails), FullScreen
  ]{hyperref}


  %%% provide all graphics (also) in this format, so you don't have
  %%% to add the file extensions to the \includegraphics-command
  %%% and/or you don't have to distinguish between generating
  %%% dvi/ps (through latex) and pdf (through pdflatex)
  \DeclareGraphicsExtensions{.pdf}

\else %else   (definitions for using latex instead of pdflatex)

  \usepackage[dvips]{graphicx}

  \DeclareGraphicsExtensions{.eps}

  \usepackage[%
    dvips,           %% sets up hyperref for use with the dvips driver
    colorlinks=false %% better for printout version; almost every hyperref-extension is eliminated by using dvips
  ]{hyperref}

\fi


%%% sets the PDF-Information options
%%% (see fields in Acrobat Reader: ``File -> Document properties -> Summary'')
%%% Note: this method is better than as options of the hyperref-package (options are expanded correctly)
\hypersetup{
  pdftitle={}, %%
  pdfauthor={Mario Murrent}, %%
  pdfsubject={}, %%
  pdfcreator={Accomplished with LaTeX2e and pdfLaTeX with hyperref-package.}, %% 
  pdfproducer={}, %%
  pdfkeywords={} %%
}


%%%%%%%%%%%%%%%%%%%%%%%%%%%%%%%%%%%%%%%%%%%%%%%%%%%%%%%%%%%%%%%%%%%%%%%%%%%%%%%%
%%%
%%% user defined commands
%%%

%%% \mygraphics{}{}{}
%% usage:   \mygraphics{width}{filename_without_extension}{caption}
%% example: \mygraphics{0.7\textwidth}{rolling_grandma}{This is my grandmother on inlinescates}
%% requires: package graphicx
%% provides: including centered pictures/graphics with a boldfaced caption below
%% 
\newcommand{\mygraphics}[3]{
  \begin{center}
    \includegraphics[width=#1, keepaspectratio=true]{#2} \\
    \textbf{#3}
  \end{center}
}

%%%%%%%%%%%%%%%%%%%%%%%%%%%%%%%%%%%%%%%%%%%%%%%%%%%%%%%%%%%%%%%%%%%%%%%%%%%%%%%%
%%%
%%% define the titlepage
%%%

% \subject{}   %% subject which appears above titlehead
% \titlehead{} %% special heading for the titlepage

%%% title
\title{An Overview of the File Transfer Protocol (FTP)}

%%% author(s)
\author{Mario Murrent}

%%% date
\date{Wiener Neustadt, am \today{}}

% \publishers{}

% \thanks{} %% use it instead of footnotes (only on titlepage)

% \dedication{} %% generates a dedication-page after titlepage


%%% uncomment following lines, if you want to:
%%% reuse the maketitle-entries for hyperref-setup
%\newcommand\org@maketitle{}
%\let\org@maketitle\maketitle
%\def\maketitle{%
%  \hypersetup{
%    pdftitle={\@title},
%    pdfauthor={\@author}
%    pdfsubject={\@subject}
%  }%
%  \org@maketitle
%}
\hypersetup{linkcolor = black}
\renewcommand{\contentsname}{Table of Contents}

%%%%%%%%%%%%%%%%%%%%%%%%%%%%%%%%%%%%%%%%%%%%%%%%%%%%%%%%%%%%%%%%%%%%%%%%%%%%%%%%
%%%
%%% set heading and footer
%%%

%%% scrheadings default: 
%%%      footer - middle: page number
\pagestyle{scrheadings}

%%% user specific
%%% usage:
%%% \position[heading/footer for the titlepage]{heading/footer for the rest of the document}

%%% heading - left
% \ihead[]{}

%%% heading - center
% \chead[]{}

%%% heading - right
% \ohead[]{}

%%% footer - left
% \ifoot[]{}

%%% footer - center
% \cfoot[]{}

%%% footer - right
% \ofoot[]{}



%%%%%%%%%%%%%%%%%%%%%%%%%%%%%%%%%%%%%%%%%%%%%%%%%%%%%%%%%%%%%%%%%%%%%%%%%%%%%%%%
%%%
%%% begin document
%%%

\begin{document}

% \pagenumbering{roman} %% small roman page numbers

%%% include the title
% \thispagestyle{empty}  %% no header/footer (only) on this page
 \maketitle

%%% start a new page and display the table of contents
\newpage
\tableofcontents

%%% start a new page and display the list of figures
% \newpage
% \listoffigures

%%% start a new page and display the list of tables
% \newpage
% \listoftables

%%% display the main document on a new page 
% \newpage

% \pagenumbering{arabic} %% normal page numbers (include it, if roman was used above)

%%%%%%%%%%%%%%%%%%%%%%%%%%%%%%%%%%%%%%%%%%%%%%%%%%%%%%%%%%%%%%%%%%%%%%%%%%%%%%%%
%%%
%%% begin main document
%%% structure: \section \subsection \subsubsection \paragraph \subparagraph
%%%

\newpage
\section{Overview}
The File Transfer Protocol (FTP) was one of the first efforts to create a standard means of exchanging files over a TCP/IP network. FTP was designed to be used over networks as well as being engineered to have the capability with exchanging files with a broad variety of machines.

The base specification is RFC 959 and is dated October 1985.  There are some additional RFCs relating to FTP.  The purpose of this document is to provide general information about how the protocol works without getting into too many technical details.  RFC 959 should be consulted for details on the protocol.

\section{FTP Commands}

\begin{itemize}
  \item USER (Username) \hfill\\
  The argument field is a Telnet string identifying the user. The user identification is that which is required by the server for access to its 
file system.
  \item PASS (Password)\hfill\\
  The argument field is a Telnet string specifying the user's password. This command must be immediately preceded by the user name command, and, for some sites, completes the user's identification for access control.
  \item CWD (Change Working Directory) \hfill\\
  This command allows the user to work with a different directory or dataset without altering his login or account information.
  Makes the given directory be the current directory on the remote host.
  \item QUIT \hfill\\
  This command terminates a USER and, if file transfer is not in progress, 
closes the control connection. If file transfer is in progress, the connection will remain open for result response and the server will then close it.
  \item PORT \hfill\\
  This specifies an alternate data port. There are defaults for both the client 
and server data ports, and under normal circumstances this command and its reply are not needed.
  \item PASV \hfill\\
  This command requests the server's data transfer process to "listen" on a data port (which is not its default data port) and to wait for a connection rather than initiate one upon receipt of a transfer command. The response to this command includes the host and port address this server is listening on.
  \item TYPE \hfill\\
  The argument specifies the file type.
  \item PWD (Print Working Directory) \hfill\\
  This command causes the name of the current working directory to be returned in the reply.
  \item RETR \hfill\\
  This command causes the server to transfer a copy of the file specified in pathname to the client. The status and contents of the file at the server site are unaffected.
  \item STOR \hfill\\
  This command causes the FTP server to accept the data transferred via the data connection and to store the data as a file at the FTP server. If the file specified in pathname exists at the server site, then its contents shall be replaced by the data being transferred. A new file is created at the FTP server if the file specified in pathname does not already exist.
  \item LIST \hfill\\ 
  This command causes a list of file names and file details to be sent from the FTP site
  Note that a directory listing is considered a file transfer.
  \item SYST \hfill\\
  This command is used to find out the operating system of the server.
  \item FEAT \hfill\\
  This command causes the FTP server to list all new FTP features that the server supports beyond those described in RFC 959.
  \item EPRT \hfill\\
  Specifies an extended address and port to which the server should connect.
\end{itemize}

\section{Communication}
The protocol can be thought of as interactive, because clients and servers actually have a conversation where they authenticate themselves.  In addition, the protocol specifies that the client and server do not exchange data on the conversation channel.  Instead, clients and servers negotiate how to send data files on separate connections, with one connection for each data transfer.

\section{Connection}
A critical concept in understanding FTP is that while it uses TCP like many other applications, it does not use just one TCP connection for all communication the way most protocols do. The FTP model is designed around two logical channels of communication between the server and user FTP processes:

\begin{itemize}

\item Control Connection \hfill\\ This is the main logical TCP connection that is created when an FTP session is established. It is maintained throughout the FTP session and is used only for passing control information, such as FTP commands and replies. It is not used to send files. 

\item Data Connection \hfill\\ Each time that data is sent from the server to the client or vice-versa, a distinct TCP data connection is established between them. Data is transferred over this connection. When the file transfer is complete, the connection is terminated.

\end{itemize}

\section{Failover Scenarios}
If a command is not implemented a default message is sent back to the client.





%%%
%%% end main document
%%%
%%%%%%%%%%%%%%%%%%%%%%%%%%%%%%%%%%%%%%%%%%%%%%%%%%%%%%%%%%%%%%%%%%%%%%%%%%%%%%%%

% \appendix  %% include it, if something (bibliography, index, ...) follows below

%%%%%%%%%%%%%%%%%%%%%%%%%%%%%%%%%%%%%%%%%%%%%%%%%%%%%%%%%%%%%%%%%%%%%%%%%%%%%%%%
%%%
%%% bibliography
%%%
%%% available styles: abbrv, acm, alpha, apalike, ieeetr, plain, siam, unsrt
%%%
% \bibliographystyle{plain}

%%% name of the bibliography file without .bib
%%% e.g.: literatur.bib -> \bibliography{literatur}
% \bibliography{FIXXME}

\end{document}
%%% }}}
%%% END OF FILE
%%%%%%%%%%%%%%%%%%%%%%%%%%%%%%%%%%%%%%%%%%%%%%%%%%%%%%%%%%%%%%%%%%%%%%%%%%%%%%%%
%%% Notice!
%%% This file uses the outline-mode of emacs and the foldmethod of Vim.
%%% Press 'zi' to unfold the file in Vim.
%%% See ':help folding' for more information.
%%%%%%%%%%%%%%%%%%%%%%%%%%%%%%%%%%%%%%%%%%%%%%%%%%%%%%%%%%%%%%%%%%%%%%%%%%%%%%%%
%% Local Variables:
%% mode: outline-minor
%% OPToutline-regexp: "%% .*"
%% OPTeval: (hide-body)
%% emerge-set-combine-versions-template: "%a\n%b\n"
%% End:
%% vim:foldmethod=marker
